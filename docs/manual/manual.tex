\documentclass[12pt]{myland}
\usepackage{pgfplots}

%%% Formatting
\def\<#1>{\texttt{#1}}
\setlength{\parskip}{1em}

%%%%%%%%%%%%%%%%%%%%%%%%%%%%FILE%%%%%%%%%%%%%%%%%%%%%%%%%%%%%%%%%%%%
\begin{document}

\title{\texttt{biacpype} User Manual}
\author{\small{Adcock Lab} \\ \small{Center for Cognitive Neuroscience, Duke University}}
\maketitle
\tableofcontents
\newpage

%%%%% section: Overview
\section{Overview}

    \texttt{biacpype} serves as a pipeline for converting raw fMRI data from Brain Imaging \& Analysis Center
    \href{https://www.biac.duke.edu/}{(BIAC)} at Duke University to the new standard Brain Imaging Data Structure
    \href{http://bids.neuroimaging.io/}{(BIDS)} format, and supporting furthur analysis. The main parts of
    \texttt{biacpype} contains:

    \begin{itemize}
        \item \texttt{biac2bids}: conversion from BIAC to BIDS
        \item ...
    \end{itemize}

    The structure of this repository is: \par \vspace{.2in}

    \begin{lstlisting}
|-biacpype // serves as the library
|    |
|    |_biac2bids
|    |_ ...
|
|-scripts  // scripts for running a variety of pipelines
|    |
|    |_convert_to_bids.py
|    |_ ...
|
|-docs     // documentation (under construction)
    |
    |_ ...
    \end{lstlisting}

    The following section will introduce main modules of \texttt{biacpype}, and the corresponding scripts for using
    the pipelines. \par

    If you have any questions, please email Preston Jiang at \href{mailto:linxing.jiang@duke.edu}{linxing.jiang@duke.edu}
    or \href{mailto:prestonj@cs.washington.edu}{prestonj@cs.washington.edu}  \par

%%%%% section: biac2bids
\section{\texttt{biac2bids}}

   \texttt{biac2bids} module is the pipeline for converting raw data from BIAC in forms of \texttt{bxh} and
   \texttt{nifti} to BIDS format. The workflow is as follows:

    \begin{figure}[h]
        \begin{mybox}
        \begin{tikzpicture}[>=latex,shape aspect=1,scale=0.7]
        \tikzstyle{box} = [rectangle, draw, minimum height=1cm, minimum width=2cm]
        % Disk Interface %
        \node[box, dashed] (biac) at (-6, 0) {Data from biac};
        \node[box, dashed] (json) at (0, 0) {\texttt{json} files};
        \path[->] (biac) edge node[midway, above] {\tiny{\texttt{generate\_json}}} (json);
        \node[box, dashed] (bxh) at (6, 0) {raw BIDS files};
        \path[->] (json) edge node[midway, above] {\tiny{\texttt{bxh2bids}}} (bxh);
        \node[box, dashed] (done) at (12, 0) {valid BIDS files};
        \path[->] (bxh) edge node[midway, above] {\tiny{\texttt{clean\_names}}} (done);
        \end{tikzpicture}
        \end{mybox}
    \end{figure}

    %%%%% subsection: prerequisites
    \subsection{Prerequisites}
    \<biac2bids> makes the following assumptions on the format of your raw data from BIAC:
   \begin{lstlisting}[escapeinside={(*@}{@*)}]
|-Data
|    |
|    |-Func
|    |  |
|    |  |-<[date_]subject>
|    |  |       |
|    |  |       |-<biac5_subject_task_run>.bxh
|    |  |       |-<biac5_subject_task_run>.nii.gz
|    |  |       |-...
|    |  |       |-(*@\color{red}{series\_order\_note.tsv}@*)
|    |  |-...
|    |
|    |-Anat
|       |-<[date_]subject>
|       |       |
|       |       |-<biac5_subject_task_run>.bxh
|       |       |-<biac5_subject_task_run>.nii.gz
|       |       |-...
|       |       |-(*@\color{red}{series\_order\_note.tsv}@*)
|       |-...
|
|-(*@\color{red}{biac\_id\_mapping.tsv}@*)
    \end{lstlisting}

    Explanations:
    \begin{itemize}
        \item \<Data> folder has to contain \<Func> and \<Anat>, and they must have the exact same folders
        \item Subfolders in \<Func> and \<Anat> are in format \<[date\_]subject> where \<[date\_]> is optional. E.g.
            \<19354> and \<20140101\_19354> are both acceptable.
        \item Each file in \<Func> and \<Anat> must in format \<biac5\_subject\_task\_run>. Usually, \<task> is a single
        digit number, \<run> is two-digit. E.g. \<biac5\_19354\_4\_01.bxh>
        \item Each subfolder in \<Func> and \<Anat> \textbf{must contain} a \color{red}{\<series\_order\_note.tsv>} \color{black} to tell the pipeline
        what each \<task> number stands for. E.g. 4 stands for ``TRAIN''. Requirements for this file are later explained.
        \item In the same folder containing \<Data>, there \textbf{must contain} a \color{red}{\<biac\_id\_mapping.tsv>}
        \color{black} which tells the pipeline the mapping from BIAC\_ID (e.g. 19354) to the session name (e.g. Session-1) and the
        Real\_ID used by your lab (e.g. 101). Requirements for this file are later explained.
    \end{itemize}

    Requirements on \<series\_order\_note.tsv> are as follows:
    \begin{table}[!htb]
        \centering
        \begin{tabularx}{.6\linewidth}{Y|Y}
             4 & LOCALIZER \\
             7 & TRAIN1 \\
             ... & ... \\
        \end{tabularx}
    \end{table}
    \par
    Note:
    \begin{itemize}
        \item The values must be \textbf{tab} separated.
        \item Task numbers in the first column \textbf{\em{must}} match that in the imaging files. For example, a ``4''
            in \<translation\_file> and a ``004'' in the imaging file will result in an error.
        \item The first column must be task code, and the second column must be the task name. There can only be two columns
    \end{itemize}

    Requirements on \<biac\_id\_mapping.tsv> are as follows:
    \begin{table}[!htb]
        \centering
        \begin{tabularx}{.6\linewidth}{Y|Y|Y}
             BIAC\_ID & Session & Real\_ID \\ \hline
             19354 & SRM & 101 \\
             19338 & SPM & 102 \\
             19368 & SPM & 101 \\
             ... & ... & ... \\
         \end{tabularx}
    \end{table}
    \par
    Note:
    \begin{itemize}
        \item The values must be \textbf{tab} separated.
        \item The headers must follow the rules (watch letter cases)!
        \item If your experiment does not multiple sessions, use only the first and last column.
    \end{itemize}
    %%%%% subsection: Scripts
    \subsection{Scripts}
    The script associated with this module is \<scripts/convert\_to\_bids.py>. There are four 
    parameters the user has to toggle. They are:
    \begin{itemize}
        \item \<STUDY\_PATH>: the path to your study file (which contains \<Data> and \<bids\_id\_mapping.csv>)
        \item \<JSON\_OUTPUT\_PATH>: the path where the user wants the json files to be saved
        \item \<BIDS\_PATH>: the path where the user wants the new BIDS format data to be saved
        \item \<LOG\_PATH>: the path where the user wants the logs to be saved
    \end{itemize}

\end{document}
