\documentclass[12pt]{myland}
\usepackage{pgfplots}

%%% Formatting
\def\<#1>{\texttt{#1}}
\setlength{\parskip}{1em}

%%%%%%%%%%%%%%%%%%%%%%%%%%%%FILE%%%%%%%%%%%%%%%%%%%%%%%%%%%%%%%%%%%%
\begin{document}

\title{\texttt{biacpype} User Manual}
\author{\small{Adcock Lab} \\ \small{Center for Cognitive Neuroscience, Duke University}}
\maketitle
\tableofcontents
\newpage

%%%%% section: Overview
\section{Overview}

    \texttt{biacpype} serves as a pipeline for converting raw fMRI data from Brain Imaging \& Analysis Center
    \href{https://www.biac.duke.edu/}{(BIAC)} at Duke University to the new standard Brain Imaging Data Structure
    \href{http://bids.neuroimaging.io/}{(BIDS)} format, and supporting furthur analysis. The main parts of
    \texttt{biacpype} contains:

    \begin{itemize}
        \item \texttt{util}: BIAC dataset validation, logging 
        \item \texttt{biac2bids}: conversion from BIAC to BIDS
        \item ...
    \end{itemize}

    The structure of this repository is: \par \vspace{.2in}

    \begin{lstlisting}
|-biacpype // serves as the library
|    |
|    |_biac2bids
|    |_ ...
|
|-scripts  // scripts for running a variety of pipelines
|    |
|    |_convert_to_bids.py
|    |_ ...
|-tutorial // a tutorial notebook
|
|-docs     // documentation (under construction)
    |_ ...
    \end{lstlisting}

    To install dependencies of this repo, run
    \[\text{\<pip install -r requirements.txt>}\]

    The following section will introduce main modules of \texttt{biacpype}, and the corresponding scripts for using
    the pipelines. All scripts are described in \textbf{Scripts} subsections. \par

    If you have any questions, please email Preston Jiang at 
    \href{mailto:linxing.jiang@duke.edu}{linxing.jiang@duke.edu}
    or \href{mailto:prestonj@cs.washington.edu}{prestonj@cs.washington.edu}  \par

\newpage
%%%%% section: util
\section{\texttt{util}}
This is the module which serves as the utility module. It contains validation functionality to let you validate your 
dataset from BIAC to check if it is ready to be converted. 

\subsection{Dataset Structure}
   The pipeline makes the following assumptions on the format of your raw data from BIAC:
   \begin{lstlisting}[escapeinside={(*@}{@*)}]
|-Data
|    |
|    |-Func
|    |  |
|    |  |-<[date_]subject>
|    |  |       |
|    |  |       |-<biac5_subject_task_run>.bxh
|    |  |       |-<biac5_subject_task_run>.nii.gz
|    |  |       |-...
|    |  |       |-(*@\color{red}{series\_order\_note.tsv}@*)
|    |  |-...
|    |
|    |-Anat
|       |-<[date_]subject>
|       |       |
|       |       |-<biac5_subject_task_run>.bxh
|       |       |-<biac5_subject_task_run>.nii.gz
|       |       |-...
|       |       |-(*@\color{red}{series\_order\_note.tsv}@*)
|       |-...
|
|-(*@\color{red}{biac\_id\_mapping.tsv}@*)
    \end{lstlisting}

    Explanations:
    \begin{itemize}
        \item \<Data> folder has to contain \<Func> and \<Anat>, and they must have the exact same folders
        \item Subfolders in \<Func> and \<Anat> are in format \<[date\_]subject> where \<[date\_]> is optional. E.g.
            \<19354> and \<20140101\_19354> are both acceptable.
        \item Each file in \<Func> and \<Anat> must in format \<biac5\_subject\_task\_run>. Usually, \<task> is a single
        digit number, \<run> is two-digit. E.g. \<biac5\_19354\_4\_01.bxh>
        \item Each subfolder in \<Func> and \<Anat> \textbf{must contain} a \color{red}{\<series\_order\_note.tsv>} \color{black} to tell the pipeline
        what each \<task> number stands for. E.g. 4 stands for ``TRAIN''. Requirements for this file are later explained.
        \item In the same folder containing \<Data>, there \textbf{must contain} a \color{red}{\<biac\_id\_mapping.tsv>}
        \color{black} which tells the pipeline the mapping from BIAC\_ID (e.g. 19354) to the session name (e.g. Session-1) and the
        Real\_ID used by your lab (e.g. 101). Requirements for this file are later explained.
    \end{itemize}

    Requirements on \<series\_order\_note.tsv> are as follows:
    \begin{table}[!htb]
        \centering
        \begin{tabularx}{.6\linewidth}{Y|Y}
             4 & LOCALIZER \\
             7 & TRAIN1 \\
             ... & ... \\
        \end{tabularx}
    \end{table}
    \par
    Note:
    \begin{itemize}
        \item The values must be \textbf{tab} separated.
        \item The first column serves as the primary key (they must be unique)
        \item Task numbers in the first column \textbf{\em{must}} match that in the imaging files. For example, a ``4''
            in \<translation\_file> and a ``004'' in the imaging file will result in an error.
        \item The first column must be task code, and the second column must be the task name. There can only be two columns
        \item If you have fmap data task code, be sure to name the translation \textbf{``fmap''} exactly. In this way, the data
            will be put in \<fmap> folder after BIDS conversion, not \<anat> folder.
    \end{itemize}

    Requirements on \<biac\_id\_mapping.tsv> are as follows:
    \begin{table}[!htb]
        \centering
        \begin{tabularx}{.6\linewidth}{Y|Y|Y}
            BIAC\_ID & [Session] & Real\_ID \\ \hline
             19354 & SRM & 101 \\
             19338 & SPM & 102 \\
             19368 & SPM & 101 \\
             ... & ... & ... \\
         \end{tabularx}
    \end{table}
    \par
    Note:
    \begin{itemize}
        \item The values must be \textbf{tab} separated.
        \item The first column serves as the primary key (they must be unique)
        \item The headers must follow the rules (watch letter cases)!
        \item If your experiment does not have multiple sessions, you can ignore the \<Session> column
    \end{itemize}

    %%%%% Validaiton
    \subsection{Validation}

    The requirements the module validates are:
    \begin{enumerate}
        \item Basic structure: check folder has \<Data>, \<biac\_id\_mapping.tsv>, Data has \<Anat>, \<Func>, 
            both \<Anat> and \<Func> contain the same folders, and each of these folders contains \<series\_order\_note.tsv>
        \item \<biac\_id\_mapping\_file>: check \<biac\_id\_mapping.tsv> is valid 
        \item \<all\_series\_order\_note\_files>: check all of the \<series\_order\_note.tsv> files in each subfolder is valid 
        \item Data matching: check each \<.bxh> file is matched with \<.nii.gz> file, and has valid task code
            translation which can be found in \<series\_order\_note.tsv>
    \end{enumerate}

    %%%%% script
    \subsection{Scripts}
    The script associated with this module is \<scripts/validate\_biac\_study\_folder.py>. The only thing you 
    need to change is the \<STUDY\_PATH>. \par

    If there are errors, they will be printed out to console after validation finishes. Otherwise, you will 
    see the following printed out:
    \[\text{\<Your study path passed validation! You are now ready for conversion>}\]

    You can also check out all the logs produced under \<biacpype/logs/validation.log>

    

%%%%% section: biac2bids
\section{\texttt{biac2bids}}

   \texttt{biac2bids} module is the pipeline for converting raw data from BIAC in forms of \texttt{bxh} and
   \texttt{nifti} to BIDS format. The workflow is as follows:

    \begin{figure}[h]
        \begin{mybox}
            \centering
        \begin{tikzpicture}[>=latex,shape aspect=1,scale=0.7]
        \tikzstyle{box} = [rectangle, draw, minimum height=1cm, minimum width=2cm]
        % Disk Interface %
        \node[box, dashed] (biac) at (-6, 0) {Data from biac};
        \node[box, dashed] (json) at (0, 0) {\texttt{json} files};
        \path[->] (biac) edge node[midway, above] {\tiny{\texttt{generate\_json}}} (json);
        \node[box, dashed] (bxh) at (6, 0) {raw BIDS files};
        \path[->] (json) edge node[midway, above] {\tiny{\texttt{bxh2bids}}} (bxh);
        \node[box, dashed] (done) at (12, 0) {valid BIDS files};
        \path[->] (bxh) edge node[midway, above] {\tiny{\texttt{clean\_names}}} (done);
        \end{tikzpicture}
        \end{mybox}
        \caption{\<biac2bids> workflow}
    \end{figure}

    %%%%% subsection: Scripts
    \subsection{Scripts}
    The script associated with this module is \<scripts/convert\_to\_bids.py>. There are four 
    parameters the user has to toggle. They are:
    \begin{itemize}
        \item \<STUDY\_PATH>: the path to your study file (which contains \<Data> and \<bids\_id\_mapping.csv>)
        \item \<JSON\_OUTPUT\_PATH>: the path where the user wants the json files to be saved
        \item \<BIDS\_PATH>: the path where the user wants the new BIDS format data to be saved
        \item \<LOG\_PATH>: the path where the user wants the logs to be saved
    \end{itemize}

    %%%%% common errors
    \subsection{Common Errors}
    \begin{enumerate}
        \item \<Error:root:scan description not found in template file!> \par
            This is because the scan description (printed out next line) is not saved in template file. 
            To fix this, add the description to \<biacpype/biac2bids/bxh2bids/info\_field\_files/psd\_types.json> following
            its format.
    \end{enumerate}


\section{\<helper>}
For now, the \<helper> contains one helper function which creates \<series\_order\_note.tsv> automatically from user 
script input.



\end{document}
